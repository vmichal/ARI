\documentclass[twoside]{article}
\usepackage[a4paper]{geometry}
\geometry{verbose,tmargin=2.5cm,bmargin=2cm,lmargin=2cm,rmargin=2cm}
\usepackage{fancyhdr}
\pagestyle{fancy}

% nastavení pisma a~češtiny
\usepackage{lmodern}
\usepackage[T1]{fontenc}
\usepackage[utf8]{inputenc}
\usepackage[czech]{babel}

% odkazy
\usepackage{url}

\usepackage{float}
% vícesloupcové tabulky
\usepackage{multirow}
\usepackage{amssymb}
\usepackage{bbold}
\usepackage{amsmath}
\usepackage{mathtools}
\usepackage{commath}

% vnořené popisky obrázků
\usepackage{subcaption}

% automatická konverze EPS 
\usepackage{graphicx} 
\usepackage{epstopdf}
\epstopdfsetup{update}

\graphicspath{{./images}}

% odkazy a~záložky
\usepackage[unicode=true, bookmarks=true,bookmarksnumbered=true,
bookmarksopen=false, breaklinks=false,pdfborder={0 0 0},
pdfpagemode=UseNone,backref=false,colorlinks=true] {hyperref}

% Poznámky při překladu
\usepackage{xkeyval}	% Inline todonotes
\usepackage[textsize = footnotesize]{todonotes}
\presetkeys{todonotes}{inline}{}

%https://tex.stackexchange.com/questions/2783/bold-calligraphic-typeface
\DeclareMathAlphabet\mathbfcal{OMS}{cmsy}{b}{n}

% Zacni sekci slovem ukol
\renewcommand{\thesection}{Úkol \arabic{section}}
% enumerate zacina s pismenem
\renewcommand{\theenumi}{\alph{enumi}}

% smaz aktualni page layout
\fancyhf{}
% zahlavi
\usepackage{titling}
\fancyhf[HC]{\thetitle}
\fancyhf[HLE,HRO]{\theauthor}
\fancyhf[HRE,HLO]{\today}
 %zapati
\fancyhf[FLE,FRO]{\thepage}

% údaje o autorovi
\title{Automatické řízení: DÚ 6 -- PID regulátor}
\author{Vojtěch Michal}
\date{\today}

\begin{document}

\maketitle

\section{Umístění pólů}

Pro mé datum narození 18. února 2000 vychází přenos soustavy
\begin{equation}
	F(s) = \frac{b(s)}{a(s)} = \frac{s+8} {s^2 + 2}.
\end{equation}
Pro zadaný přenos navrhněte PID regulátor, který póly výsledného systému umístí
do poloh $-1$ a~$-0.5 \pm 0.5 i$. \\
\textbf{Řešení:}
Zadané poloze pólů odpovídá charakteristický polynom
\begin{equation}
	c_{goal}(s) = s^3 + 2s^2 + 1.5 s + 0.5.
	\label{eq:goal}
\end{equation}

Přenos PID regulátoru je obecně dán vztahem
\begin{equation}
	C(s) = K_P + \frac{K_I}{s} + K_D s,
\end{equation}
jehož konstanty $K_P$, $K_I$ a~$K_D$ je potřeba naladit pro zajištění požadovaných vlastností uzavřené smyčky.
Soustava $F(s)$ je řádu 2, regulátor $C(s)$ je neryzí a~řádu 1. Dohromady bude systém třetího řádu, tři stupně volnosti
konstant regulátoru jsou proto dostatečné pro spolehlivé nastavení pólů uzavřené smyčky,
není však možné nastavit libovolné ustálené zesílení.
Převeďme přenos regulátoru na racionální funkci
\begin{equation}
	C(s) = \frac{q(s)}{p(s)} = \frac{s^2 K_D + s K_P + K_I}{s}
\end{equation}
a zařaďme jej kaskádně před řízenou soustavu pro získání open loop přenosu
\begin{equation}
	L(s) = C(s) \cdot F(s) = \frac{s^2 K_D + s K_P + K_I}{s} \cdot \frac{s+8} {s^2 + 2}.
\end{equation}
Uzavřeme-li kolem $L(s)$ jednotkou zpětnou vazbu, získáme přenos $T(s) = \frac{L(s)}{1 + L(s)}$, ze kterého nás však zajímá
pouze charakteristický polynom
\begin{equation}
	c(s) = a(s)p(s) + b(s)q(s) =\underbrace{s}_{p(s)} \cdot \underbrace{(s^2 + 2)}_{a(s)} 
	+ \underbrace{(s+8)}_{b(s)} \cdot \underbrace{(s^2 K_D + s K_P + K_I)}_{q(s)}.
\end{equation}
Charakteristický polynom roznásobme
\begin{equation}
	\begin{split}
		c(s) &= \underbrace{s^3 + 2s}_{p(s)\cdot a(s)} + \underbrace{s^3 K_D + s^2 K_P + s K_I}_{s\cdot q(s)} +\underbrace{ 8 s^2 K_D + 8s K_P +8 K_I}_{8 q(s)} \\
		c(s) &= s^3 \cdot (1 + K_D) + s^2 \cdot(K_P + 8K_D) + s \cdot(2 + K_I + 8K_P) + 1 \cdot (8K_I)
	\end{split}
\end{equation}
a koeficient $\beta = 1 + K_D$ u~kubického členu vytkneme z polynomu.
\begin{equation}
	c(s) = (1 + K_D) \cdot( s^3 + s^2 \cdot \frac{K_P + 8K_D}{1 + K_D} + s \cdot  \frac{2 + K_I + 8K_P}{1 + K_D} + 1 \cdot \frac{8K_I}{1 + K_D})
\end{equation}
Porovnáme s žádaným polynomem dle rovnice \eqref{eq:goal}. Identita $c(s) = c_{goal}(s)$ musí být splněna po jednotlivých koeficientech.
Vzniká soustava tří rovnic
\begin{equation}
	\begin{split}
		\frac{K_P + 8K_D}{1 + K_D} &= 2 \\
		\frac{2 + K_I + 8K_P}{1 + K_D} &= 1.5\\
		\frac{8K_I}{1 + K_D} &= 0.5.
	\end{split}
\end{equation}
Po převedení na lineární rovnice
\begin{equation}
	\begin{split}
		K_P + 8K_D &= 2 + 2K_D \\
		2 + K_I + 8K_P &= 1.5 + 1.5 K_D\\
		8K_I &= 0.5 +0.5 K_D,
	\end{split}
\end{equation}
přeuspořádání
\begin{equation}
	\begin{split}
		6K_D + K_P &= 2 \\
		- 1.5 K_D + 8K_P  + K_I &= -0.5\\
		- 0.5K_D + 8K_I &= 0.5,
	\end{split}
\end{equation}
převedení na maticový tvar
\begin{equation}
	\begin{bmatrix}
		6 & 1 & 0 \\
		-1.5 & 8 & 1 \\
		-0.5 & 0 &8  
		
	\end{bmatrix} \cdot \begin{bmatrix}
		K_D \\
		K_P \\
		K_I
	\end{bmatrix} = \begin{bmatrix}
		2 \\
		-0.5 \\
		0.5
	\end{bmatrix}
\end{equation}
a vyřešení soustavy Matlabem vyjde 
\begin{equation}
	\begin{bmatrix}
		K_D \\
		K_P \\
		K_I
	\end{bmatrix} = 
	\begin{bmatrix}
		0.3350 \\
		-0.0101 \\
		0.0834
	\end{bmatrix}.
\end{equation} \\
\textbf{Závěr:}
Žádaný PID regulátor má přenos
\begin{equation}
	C(s) = -0.0101 + \frac{0.0834}{s} + 0.3350 s.
\end{equation}
Protože má úloha pouze tři stupně volnosti, podařilo se umístit všechny tři póly, nelze však nastavit ustálené zesílení
uzavřené zpětnovazební smyčky. Pro jeho doladění by bylo nezbytné přidat další stupeň volnosti, například
celé smyčce předřadit konstantní skalární zesílení.

\begin{thebibliography}{9}

\bibitem{motivace}
	Robert H. Bishop, Supplementary lectures to book \emph{Modern control systems, 13th edition} \url{https://www.youtube.com/watch?v=WaM8_Bllaf0}

\end{thebibliography}












\end{document}

