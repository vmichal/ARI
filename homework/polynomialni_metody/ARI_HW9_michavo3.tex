\documentclass[twoside]{article}
\usepackage[a4paper]{geometry}
\geometry{verbose,tmargin=2.5cm,bmargin=2cm,lmargin=2cm,rmargin=2cm}
\usepackage{fancyhdr}
\pagestyle{fancy}

% nastavení pisma a~češtiny
\usepackage{lmodern}
\usepackage[T1]{fontenc}
\usepackage[utf8]{inputenc}
\usepackage[czech]{babel}

% odkazy
\usepackage{url}

\usepackage{float}
% vícesloupcové tabulky
\usepackage{multirow}
\usepackage{amssymb}
\usepackage{gensymb}
\usepackage{bbold}
\usepackage{amsmath}
\usepackage{mathtools}
\usepackage{commath}

% vnořené popisky obrázků
\usepackage{subcaption}

% automatická konverze EPS 
\usepackage{graphicx} 
\usepackage{epstopdf}
\epstopdfsetup{update}

\graphicspath{{./images}}

% odkazy a~záložky
\usepackage[unicode=true, bookmarks=true,bookmarksnumbered=true,
bookmarksopen=false, breaklinks=false,pdfborder={0 0 0},
pdfpagemode=UseNone,backref=false,colorlinks=true] {hyperref}

% Poznámky při překladu
\usepackage{xkeyval}	% Inline todonotes
\usepackage[textsize = footnotesize]{todonotes}
\presetkeys{todonotes}{inline}{}

%https://tex.stackexchange.com/questions/2783/bold-calligraphic-typeface
\DeclareMathAlphabet\mathbfcal{OMS}{cmsy}{b}{n}

% Zacni sekci slovem ukol
\renewcommand{\thesection}{Úkol \arabic{section}}
% enumerate zacina s pismenem
\renewcommand{\theenumi}{\alph{enumi}}

% smaz aktualni page layout
\fancyhf{}
% zahlavi
\usepackage{titling}
\fancyhf[HC]{\thetitle}
\fancyhf[HLE,HRO]{\theauthor}
\fancyhf[HRE,HLO]{\today}
 %zapati
\fancyhf[FLE,FRO]{\thepage}

% údaje o autorovi
\title{Automatické řízení: DÚ 9 -- Polynomiální metody}
\author{Vojtěch Michal}
\date{\today}

\begin{document}

\maketitle

Jsem narozen 18. února 2000. Proto volím $d = 1$, $r = 2$, $m = 2$.

\section{Diofantická rovnice}

Řešte polynomiální rovnici
\begin{equation}
	\label{eq:zadani}
	\underbrace{(2 + 3s + s^2)}_{=a(s)}x(s) + \underbrace{(1 + s)}_{=b(s)}y(s) = \underbrace{5 + 8s + 4s^2 + s^3}_{=c(s)}
\end{equation}
\begin{itemize}
	\item Rovnici řešte pomocí Sylvestrovy matice a elementárními operacemi. Najděte obecné řešení.
	\item Najděte řešení minimálního stupně v x.
	\item Najděte řešení minimálního stupně v y.
	\item Nakonec řešte přímo funkcí \textit{axbyc} Polynomial Toolboxu, a to pro všechny typy řešení
	minimálních stupňů.
\end{itemize}

\textbf{Řešení:}
Jsou dány polynomy
\begin{equation}
	\begin{align}
		a(s) &= s^2 + 3s + 2 = (s+1)(s+2), \\
		b(s) &= s + 1, \\
		c(s) &= s^3 + 4s^2 + 8s + 5 = (s+1)(s^2 + 3s + 5).
	\end{align}
\end{equation} \\
Bude se mi hodit znát hodnotu
\begin{equation}
	g(s) = \text{gcd}(a(s), b(s)) = s + 1
\end{equation}
a od něj odvozených
\begin{equation}
	\begin{align}
		\bar{a}(s) &= \frac{a(s)}{g(s)} = s + 2 \\
		\bar{b}(s) &= \frac{b(s)}{g(s)} = 1 \\
	\end{align}
\end{equation}

\subsection{Řešitelnost}
Diofantická rovnice v polynomech bude mít řešení právě tehdy, když $\text{gcd}(a(s), b(s))$ dělí beze zbytku $c(s)$. Protože polynom $c(s)$ má kořen -1,
je beze zbytku dělitelný polynomem $g(s)$ a řešení Diofantické rovnice tedy má smysl hledat.

\subsection{Sylvestrova matice}
Pro řešení polynomiální Diofantické rovnice
\begin{equation}
	a(s)x(s) + b(s)y(s) = c(s)
\end{equation}
je potřeba odhadnout stupeň polynomů $x(s)$, $y(s)$ a následně sestavit Sylvestrovu matici $\mathbf{S}$.

Protože $\text{deg }c(s)$ = 3 a min(deg $a(s)$, deg $b(s)$) = 1, odhaduji  $x(s) = x_2 s^2 + x_1 s + x_0$,  $y(s) = y_2 s^2 + y_1 s + y_0$.
Odtud se problém řešení Diofantické rovnice převádí na řešení rovnice
\begin{equation}
	\begin{bmatrix}
		x_0 & y_0 & x_1 & y_1 & x_2 & y_2
	\end{bmatrix}
	\underbrace{\begin{bmatrix}
		2  &   3  &   1  &   0   &  0 \\
		1  &   1  &   0  &   0   &  0 \\
		0  &   2  &   3  &   1   &  0 \\
		0  &   1  &   1  &   0   &  0 \\
		0  &   0  &   2  &   3   &  1 \\
		0   &  0   &  1   &  1    & 0
	\end{bmatrix}}_{=S} = \begin{bmatrix}
		5 & 8 & 4 &1 & 0
	\end{bmatrix}.
\end{equation}

Z posledního sloupce matice je patrné, že $x_2 = 0$. Rozepsání zbytku soustavy na jednotlivé rovnice
pro dílčí koeficienty polynomů $x(s)$, $y(s)$ dává rovnice
\begin{equation}
	\begin{align}
		2x_0 + y_0 &= 5 \\
		3x_0 + y_0 + 2x_1 + y_1 &= 8 \\
		x_0 + 3x_1 + y_1 + y_2 &= 4 \\
		x_1 + y_2 &= 1 \\
	\end{align}
\end{equation}

\subsection{Řešení minimální v x}
\textit{Řešení minimální v x} je takové, pro které platí deg $x(s)$ < deg $\bar{b}(s)$. Protože je ale $\bar{b}(s)$
polynom stupně nula (je to nenulová konstanta), plyne rovnou $x(s) = 0$, protože neexistuje jiný polynom se stupněm nižším než nula.

\subsection{Řešení minimální v y}
\textit{Řešení minimální v x} je takové, pro které platí deg $y(s)$ < deg $\bar{a}(s)$. Platí deg $\bar{a}(s) = 1$,
proto hledám polynom $y(s)$ jako nenulovou konstantu. Zadaná rovnice $ax + by = c$ má obecné řešení
\begin{equation}
	\begin{align}
		x &= x_p - \bar{b} \cdot t, \\
		y &= y_p + \bar{a} \cdot t,
	\end{align}
\end{equation}
kde $t(s)$ je libovolný polynomiální parametr.

\subsection{Řešení funkcí \textit{axbyc}}
Volání Matlabovské funkce \textit{axbyc} s polynomy $a(s)$, $b(s)$, $c(s)$ vrací následující výsledky:
\begin{itemize}
	\item Pro \textit{axbyc}(a, b, c,\textbf{'minx'}) jsou nalezené polynomy $x(s) = 0$, $y(s) = s^2 + 3s + 5$.
	\item Pro \textit{axbyc}(a, b, c,\textbf{'miny'}) jsou nalezené polynomy $x(s) = s+1$, $y(s) = 3$.
\end{itemize}

\section{Asymptotické sledování}
Pro soustavu s přenosem
\begin{equation}
	\label{eq:prenos}
	G(s) = \frac{b(s)}{a(s)} = \frac{s+2}{s(s-1)}
\end{equation}

\begin{itemize}
	\item  Navrhněte polynomiálními metodami regulátor se dvěma stupni volnosti, který zajistí asymptotické
	sledování ramp. Póly výsledného systému umístěte do poloh $s_{1,2} = -2 \pm j$. Bude-li potřeba 3. pól, vykraťte
	jím nulu soustavy.
 	\item Vypočtěte přenos výsledného systému z reference na výstup soustavy. Překontrolujte, zda odpovídá
	požadavkům.
	\item Vykreslete odezvu výsledného systému na skok reference a na rampu reference
\end{itemize}

\end{document}

