\documentclass[twoside]{article}
\usepackage[a4paper]{geometry}
\geometry{verbose,tmargin=2.5cm,bmargin=2cm,lmargin=2cm,rmargin=2cm}
\usepackage{fancyhdr}
\pagestyle{fancy}

% nastavení pisma a češtiny
\usepackage{lmodern}
\usepackage[T1]{fontenc}
\usepackage[utf8]{inputenc}
\usepackage[czech]{babel}

% odkazy
\usepackage{url}

\usepackage{float}
% vícesloupcové tabulky
\usepackage{multirow}
\usepackage{amssymb}
\usepackage{bbold}
\usepackage{amsmath}
\usepackage{mathtools}
\usepackage{commath}

% vnořené popisky obrázků
\usepackage{subcaption}

% automatická konverze EPS 
\usepackage{graphicx} 
\usepackage{epstopdf}
\epstopdfsetup{update}

\graphicspath{{./images}}

% odkazy a záložky
\usepackage[unicode=true, bookmarks=true,bookmarksnumbered=true,
bookmarksopen=false, breaklinks=false,pdfborder={0 0 0},
pdfpagemode=UseNone,backref=false,colorlinks=true] {hyperref}

% Poznámky při překladu
\usepackage{xkeyval}	% Inline todonotes
\usepackage[textsize = footnotesize]{todonotes}
\presetkeys{todonotes}{inline}{}

%https://tex.stackexchange.com/questions/2783/bold-calligraphic-typeface
\DeclareMathAlphabet\mathbfcal{OMS}{cmsy}{b}{n}

% Zacni sekci slovem ukol
\renewcommand{\thesection}{Úkol \arabic{section}}
% enumerate zacina s pismenem
\renewcommand{\theenumi}{\alph{enumi}}

% smaz aktualni page layout
\fancyhf{}
% zahlavi
\usepackage{titling}
\fancyhf[HC]{\thetitle}
\fancyhf[HLE,HRO]{\theauthor}
\fancyhf[HRE,HLO]{\today}
 %zapati
\fancyhf[FLE,FRO]{\thepage}

% údaje o autorovi
\title{Automatické řízení: DÚ 5 -- Zpětná a přímá vazba}
\author{Vojtěch Michal}
\date{\today}

\begin{document}

\maketitle

Obrázky se schématy systémů jsou v zadání dostupném na adrese \cite{zadani}.

\section{Zpětná vazba}
Porovnejte dva systémy, jejichž schémata jsou na obrázcích v zadání.
Pro jednoznačnost označím horní systém písmenem A, dolní systém písmenem B.
Zjistěte, jaký mezi nimi může být rozdíl z hlediska stability. Rada: Najděte podmínky stability každého
z nich, porovnejte je, a případné rozdíly vysvětlete.

\textbf{Řešení:}
O stabilitě systému rozhoduje jeho charakteristický polynom. Nalezněme přenosy pro systém A
\begin{equation}
	G_A(s) = \frac{r(s)}{p(s)} \cdot \frac{\frac{b(s)}{a(s)} }{1 + \frac{b(s)}{a(s)} \cdot \frac{q(s)}{p(s)}} =
	\frac{r(s)}{p(s)} \cdot \frac{b(s) \cdot p(s)}{p(s) \cdot a(s) + b(s) \cdot q(s)}
\end{equation}
a pro systém B
\begin{equation}
	G_B(s) = r(s) \cdot \frac{\frac{b(s)}{a(s)} \cdot \frac{1}{p(s)}}{1 + \frac{b(s)}{a(s)} \cdot \frac{1}{p(s)} \cdot \frac{q(s)}{1}} =
	r(s) \cdot \frac{b(s) }{p(s) \cdot a(s) + b(s) \cdot q(s)}.
\end{equation}
Pakliže má být systém stabilní (studujeme vnitřní stabilitu, nikoli jen BIBO), poté musí být stabilní oba jeho subsystémy -- předfiltr
$\frac{r(s)}{p(s)}$ i za ním v serii zapojená zpětnovazební smyčka. Na první pohled je vidět, že rozdíl systémů je v poloze
polynomu $p(s)$. V systému A se nalézá ve jmenovateli předfiltru, v systému B se nalézá ve jmenovateli přímé větve zpětnovazební
smyčky. Uzavření zpětné vazby vytváří v principu nový systém; pomocí FB je možné stabilizovat nestabilní soustavy.
Naopak seriové řazení systémů tuto vlastnost nemá. Jakmile je systém jednou nestabilní, není možné jej stabilizovat pouze 
seriovým připojením dalšího systému (krácení pólů povede na skryté módy, které budou vybuditelné například odlišnými
počátečními podmínkami).

Matematicky lze tuto myšlenku podložit analýzou charakteristických polynomů
\begin{equation}
	\begin{split}
		c_B(s) &= p(s) \cdot a(s) + b(s) \cdot q(s), \\
		c_A(s) &= p(s) \cdot (p(s) \cdot a(s) + b(s) \cdot q(s)) = p(s) \cdot c_B(s).
	\end{split}
\end{equation}
Pro stabilitu systému B je postačující, aby měly kořeny polynomu $c_B(s)$ zápornou reálnou část.
Pro stabilitu systému A je nezbytné totéž, nad to je ale ještě potřeba, aby samotné $p(s)$ bylo stabilní,
protože se nalézá ve jmenovateli samostatně.

\textbf{Závěr:} Uzavření zpětnovazební smyčky je schopno "posunout" kořeny polynomu $p(s)$
do stabilní oblasti. Jakmile však kolem systému FB uzavřen není, je nezbytné, aby byl sám o sobě stabilní.
Oba systémy budou mít z hlediska vnějšího modelu stejný přenos, protože $p(s)$ v čitateli a jmenovateli přenosu $G_A(s)$
se proti sobě pokrátí, vnitřně však může nastat nestabilita a signál na výstupu předfiltru $\frac{r(s)}{p(s)}$ poroste nade všechny meze.

\section{Přímá vazba}
Chování systému s přenosem $P(s) = P_1(s) P_2(s)$, kde
\begin{equation}
	P_1(s) = \frac{s+2}{s+1}, ~~~ P_2(s) = \frac{1}{s-1}
\end{equation}
ovlivňuje porucha, která přichází dovnitř soustavy dle obrázku v zadání.

Naštěstí tuto poruchu můžete před vstupem do soustavy měřit. Navrhněte přímovazební a
zpětnovazební část regulátoru (tedy přenosy $F(s)$ a $C(s)$ dle obrázku) tak, aby porucha co nejméně
ovlivňovala výstup soustavy a aby celý systém byl stabilní.
Rada: Nejprve vypočtete přenos poruchy na výstup soustavy. V tomto zvláštním případě lze tento přenos
velmi vhodně upravit jednoduchou volbou přímovazebního regulátoru. Potom navrhněte stabilizující
zpětnovazební regulátor. 

\textbf{Řešení:}
Odvoďme přenos z poruchy $d(s)$ na výstup systému $y(s)$ (pro lepší čitelnost vynechávám závorky $(s)$ u přenosů $P_1$, $P_2$, $C$ a $F$
a signálů $d$, $y_r$ a $y$) Protože platí
\begin{equation}
	\begin{split}
		y = P_2 (d + P_1(C (-y) - F d)) &= P_2 d - P_2 P_1 C y - P_2 P_1 F d\\
		(1 + P_2 P_1 C)y &= P_2(1-P_1 F) d \\
		y &= \frac{P_2(1-P_1 F)}{1 + P_2 P_1 C} d,
	\end{split}
\end{equation}
je přenos z poruchy na výstup soustavy roven
\begin{equation}
	G(s) = \frac{y(s)}{d(s)} = \frac{P_2(1-P_1 F)}{1 + P_2 P_1 C}.
	\label{eq:prenos}
\end{equation}
\subsection{Stabilita}
Pro zajištění stability je nezbytné, aby kořeny charakteristického polynomu měly
reálné části záporné. Póly přenosu z rovnice \eqref{eq:prenos} jsou takové hodnoty $s$, pro které platí $1 + P_2 P_1 C = 0$. 
Použijme PI regulátor $C(s) = Kp + \frac{K_i}{s} = \frac{K_p s + K_i}{s}$, poté
\begin{equation}
	\begin{split}
		1 + C \cdot \frac{s+2}{s+1} \frac{1}{s-1} &= 0 \\
		(s+1)(s-1)s + (K_p s + K_i) (s+2) &= 0. \\
	\end{split}
\end{equation}
Pro zbytek návrhu použiji nástroj \textit{rltool} v Matlabu


\begin{thebibliography}{9}

	\bibitem{zadani} Zadání domácího úkolu \url{http://www.polyx.com/_ari/du/du_05.pdf}

\bibitem{motivace}
	Robert H. Bishop, Supplementary lectures to book \emph{Modern control systems, 13th edition} \url{https://www.youtube.com/watch?v=w0AOGeqOnFY}

\end{thebibliography}












\end{document}

