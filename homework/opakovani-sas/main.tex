\documentclass[twoside]{article}
\usepackage[a4paper]{geometry}
\geometry{verbose,tmargin=2.5cm,bmargin=2cm,lmargin=2cm,rmargin=2cm}
\usepackage{fancyhdr}
\pagestyle{fancy}

% nastavení pisma a češtiny
\usepackage{lmodern}
\usepackage[T1]{fontenc}
\usepackage[utf8]{inputenc}
\usepackage[czech]{babel}

% odkazy
\usepackage{url}

% vícesloupcové tabulky
\usepackage{multirow}
\usepackage{amssymb}
\usepackage{bbold}
\usepackage{amsmath}

% vnořené popisky obrázků
\usepackage{subcaption}

% automatická konverze EPS 
\usepackage{graphicx} 
\usepackage{epstopdf}

\graphicspath{{./images/}}

% odkazy a záložky
\usepackage[unicode=true, bookmarks=true,bookmarksnumbered=true,
bookmarksopen=false, breaklinks=false,pdfborder={0 0 0},
pdfpagemode=UseNone,backref=false,colorlinks=true] {hyperref}

% Poznámky při překladu
\usepackage{xkeyval}	% Inline todonotes
\usepackage[textsize = footnotesize]{todonotes}
\presetkeys{todonotes}{inline}{}

\newcommand{\dif}{\text{d}}

% Zacni sekci slovem ukol
\renewcommand{\thesection}{Úkol \arabic{section}}
% enumerate zacina s pismenem
\renewcommand{\theenumi}{\alph{enumi}}

% smaz aktualni page layout
\fancyhf{}
% zahlavi
\usepackage{titling}
\fancyhf[HC]{\thetitle}
\fancyhf[HLE,HRO]{\theauthor}
\fancyhf[HRE,HLO]{\today}
 %zapati
\fancyhf[FLE,FRO]{\thepage}

% údaje o autorovi
\title{Domácí úkol 0 - opakování předmětu SAS}

\author{Vojtěch Michal}
\date{\today}

\begin{document}

\maketitle

% ---------------------------------
% ---------------------------------
% název sekce je generován automaticky jako: Úkol X
\section{Ustálená odezva}
\label{sec:ukol1}

\subsection{~}
Nalezněte statické zesílení K systému popsaného níže uvedeným přenosem.
\begin{equation}
	G(s) = \frac{-0.2s+5}{(s+3)(s+2)(s+1)}
\end{equation}
Řešení: Statické zesílení je konečná hodnota odezvy na jednotkový skok $w(t)$. Podle věty o konečné hodnotě
\begin{equation}
	\lim_{t \to \infty}w(t) = \lim_{s \to 0}sW(s)
\end{equation}
pro statické zesílení K plyne:
\begin{equation}
	\label{eq:dcgain}
	K = \lim_{t \to \infty}w(t) = \lim_{s \to 0}{\underbrace{sW(s)}_{=G(s)}} = \frac{0+5}{(0+3)(0+2)(0+1)} = \frac{5}{6}.
\end{equation}
Statické zesílení zadaného vnějšího modelu je $K = \frac{5}{6}$.

\subsection{~}
Zjistěte ustálenou hodnotu odezvy na vstup $u (t) = 5$. Své tvrzení odvoďte.

Řešení: Nejdříve je potřeba ověřit, že nějaká ustálená hodnota bude existovat. Zřejmě ano, protože přenosová funkce má všechny póly
v levé polorovině s. Využijeme linearity systému a Laplaceovy transformace. Protože $u(t) = 5\cdot \mathbb{1}(t)$, platí pro výstup $y(t)$:
\begin{equation}
	y(t) = \mathcal{L}^{-1}\{G(s) \cdot \mathcal{L}\{u(t)\}\} = 5~\mathcal{L}^{-1}\{\underbrace{G(s) \cdot \mathcal{L}\{\mathbb{1}(t)\}}_{= \frac{1}{s}G(s)=W(s)}\}
	= 5 w(t) \longrightarrow_{t \to \infty} 5 K.
\end{equation}
Se znalostí statického zesílení podle \eqref{eq:dcgain} je vidět, že ustálený výstup má hodnotu $y(t)=\frac{25}{6}$.


\subsection{~}
\label{sec:ukol1:3}
Jaká bude ustálená hodnota odezvy na Dirakův impulz?

Řešení: Protože odezvy na jednotkový skok $w(t)$ a na Diracův impuls $h(t)$ spojuje vztah $w(t) = \int h(t)$ a již víme, že $w(t)$ se ustálí
na konstantní hodnotě $K$, poté její derivace musí být po ustálení nulová. Jinými slovy pro ustálenou hodnotu impulsové odezvy platí $\lim_{t \to \infty} h(t) = 0$.

% ---------------------------------
\section{Laplaceova transformace}
\label{sec:ukol2}
\subsection{~}
Pomocí Laplaceovy transformace vyřešte následující soustavu diferenciálních rovnic
\begin{equation}
	\begin{split}
		\frac{\dif x_1(t)}{\dif t} &= - 6x_1 (t) + 26x_2 (t), \\
		\frac{\dif x_2(t)}{\dif t} &= - 0.5 x_1(t),
	\end{split}
\end{equation}
za počátečních podmínek
\begin{equation}
	\vec{x}_0 = \begin{bmatrix}
		2 \\
		0
	  \end{bmatrix}
\end{equation}
Výsledkem budou funkce popisující časový průběh stavu $x_1(t)$ a $x_2(t)$. Řešte ručně a do řešení uveďte postup.

\subsection{~}
Vykreslete průběhy nalezených funkcí v Matlabu. Nezapomeňte na popisky os, titulek, legendu a přehlednost - tedy vše,
co by se v mělo u Vašich grafů v budoucnu vždy objevit.

\section{Linearizace}
\label{sec:ukol3}
Je zadaný systém
\begin{equation}
	\begin{split}
		\dot{x_1} &= x_2 \\
		\dot{x_2} &= -2 \text{sin}(x_1) - 0.1x_2 +u \\
		y &= x_1
	\end{split}
\end{equation}

\subsection{~}
Nalezněte rovnovážný pracovní bod systému $P = \begin{bmatrix} x_{1p}, x_{2p}, y_p \end{bmatrix}$ pro $u_p (t) = 2$.

\subsection{~}
Linearizujte systém v nalezeném pracovním bodě $P$.

\subsection{~}
Namodelujte v Simulinku nelineární systém současně s linearizovaným. Porovnejte odezvy obou systémů
na skok $u(t) = 1$ a $u(t) = 2.001$. Průběhy nelineárního a linearizovaného systému vykreslete do jednoho
obrázku pomocí Matlabu.

\section{Asymptotické frekvenční charakteristiky}
\label{sec:ukol4}

\subsection{~}
Pro následující systém nakreslete asymptotickou frekvenční charakteristiku.
\begin{equation}
	G_1(s) = \frac{s-1}{(s+10)(s-10)(s+80)}
\end{equation}
Jako řešení můžete vložit oskenovaný/ dobře vyfocený obrázek nebo můžete využít grafický editor. Obrázek
simulované frekvenční charakteristiky z příkazu bode v Matlabu použijte jako kontrolu, nicméně jako řešení
ho neodevzdávejte - nejedná se o asymptotické frekvenční charakteristiky.


\subsection{~}
Pro asymptotickou frekvenční charakteristiku na obrázku (2) sestavte rovnici odpovídajícího systému
s nejnižším možným počtem pólů.
\includegraphics{zadani4-2.png}

\section{ Převod do přenosového popisu}
\label{sec:ukol5}
Převeďte následující systém do přenosového popisu. Napište postup, kterým jste k výsledku dospěli, a vybrané mezivýsledky.

\begin{math}
	\dot{\vec{x}} = \begin{bmatrix}
		-4 & 2 & 0 & 0 \\
		-6 & 4 & 0 & 0 \\
		-3 & 3 & 2 & -2 \\
		-9 & 9 & 2 & -3
	\end{bmatrix} \vec{x} + \begin{bmatrix}
		1 \\
		0.5 \\
	0 \\
	-1
\end{bmatrix} u ~~~~~~~~~~~
y = \begin{bmatrix}
	4 & 0 & 0 & 0
\end{bmatrix} \vec{x}
\end{math}

\section{Stabilita}
\label{sec:ukol6}
Rozhodněte o stabilitě systému. Výsledek zdůvodněte.

\begin{math}
	\dot{\vec{x}} = \begin{bmatrix}
		1 & 0 & 0 \\
		2 & -2 & -2 \\
		1 & 2 & 0
	\end{bmatrix} \vec{x} + \begin{bmatrix}
		1 \\
		-1 \\
		1
\end{bmatrix} u ~~~~~~~~~~~
y = \begin{bmatrix}
	1 & 0 & -1
\end{bmatrix} \vec{x}
\end{math}


\section{Diskretizace}
\label{sec:ukol7}
Následující systém převeďte do diskrétního popisu, můžete využít např. metodu \textit{zero order hold}.
\begin{equation}
	G(s) = \frac{s-3}{(s+1)(s+7)}
\end{equation}
K diskretizaci můžete využít Matlab. Následně v Simulinku odsimulujte odezvy diskrétního a spojitého systému
pro dvě různé periody vzorkování - pro jednu vhodně zvolenou (dostatečně krátkou) a jednu nevhodně zvolenou.
Do jednoho grafu vykreslete výstup spojitého systému a výstup diskretizovaného systému s vhodně zvolenou
periodou vzorkování. (1b.) Druhý graf obdobně pro nevhodnou periodu vzorkování. (2b.) Jako vstup zvolte
libovolný periodický signál, např. sinus, pila, atd. Nezapomeňte na to, jak mají vypadat výsledné grafy

\section{Vlastnosti přenosů}
\label{sec:ukol8}
Rozhodněte u každého z následujích přenosů, zdali jím popsaný systém je: stabilní/nestabilní, statický/astatický, kmitavý/nekmitavý.
\begin{equation*}
	\begin{split}
		G_1(s) &= -\frac{s-12}{(s+1)(5s+2)(s+3)} \\
		G_2(s) &= \frac{1}{s^2 + 0.5s - 1} \\
		G_3(s) &= \frac{1}{s^2} \\
		G_4(s) &= \frac{s+2}{s^2-2}
	\end{split}
\end{equation*}

\section{Spojování dynamických systémů}
\label{sec:ukol9}
Jsou dány dva systémy
\begin{equation*}
	G_1(s) = \frac{s-1}{s+1}, G_2(s) = \frac{1}{s}
\end{equation*}
odvoďte výsledný přenost $H$ pro jejich ...
\begin{figure}[htbp]
	\centering
	\includegraphics[width=.4\textwidth]{zadani9-3.png}
	\label{fig:zadani9-3}
  \end{figure}
  \begin{figure}[htbp]
	\centering
	\includegraphics[width=.4\textwidth]{zadani9-4.png}
	\label{fig:zadani9-4}
  \end{figure}
\subsection{~}
paralelní zapojení
\subsection{~}
seriové zapojení
\subsection{~}
zpětnovazební spojení se zápornou zpětnou vazbou, kde přenosy G1 a G2 jsou v přímé vazbě (Odvoďte přenos H (s) = y(s)/r(s)) (viz obr \ref{fig:zadani9-3})
\subsection{~}
zpětnovazební spojení se zápornou zpětnou vazbou s G1 v přímé vazbě a G2 ve zpětné vazbě (Odvoďte přenos H (s) = y(s)/r(s)) (viz obr \ref{fig:zadani9-4})

\section{Diskrétní systém}
\label{sec:ukol10}
Je zadán diskrétní systém
\begin{equation}
	\label{eq:diferencni}
	3y(k) - y(k-1) + 0.5 y(k-2) = u(k-1)-u(k-2)
\end{equation}

\subsection{~}
Vyjádřete přenos systému. \\
Řešení: Použiji Z-transformaci diferenční rovnice popisující systém. Nechť $Y(z) = \mathcal{Z}\{y(k)\}$ a $U(z) = \mathcal{Z}\{u(k)\}$, poté \eqref{eq:diferencni} upravíme na
\begin{equation}
	\begin{split}
		3Y(z) - z^{-1} Y(z) + 0.5 z^{-2} Y(z) &= z^{-1}U(z) - z^{-2}U(z) \\
		Y(z) (3- z^{-1} + 0.5 z^{-2}) &= U(z) (z^{-1} - z^{-2}) \\
		H(z) = \frac{Y(z)}{U(z)} &= \frac{z^{-1} - z^{-2}}{3-z^{-1}+ 0.5z^{-2}}, \text{rozšiřme členem $z^2$} \\
		H(z) &= \frac{z - 1}{3z^2-z+ 0.5}.
	\end{split}
\end{equation}

\subsection{~}
Rozhodněte o stabilitě systému. Rozhodnutí zdůvodněte.
Řešení: Přenosová funkce má dva komplexně sdružené póly $ z_{1,2} = 0.1667 \pm 0.3727i$. Aby byl stabilní, musí ležet póly uvnitř jednotkové kružnice v z-rovině.
Tento požadavek je zde zřejmě splněn, protože $\vert z_{1,2} \vert < 1 $.
\subsection{~}
Nakreslete simulinkové schéma realizace tohoto přenosu pro vstup u a výstup y a vykreslete jeho odezvu
na jednotkový skok pro vzorkovací periodu h = 1


\section{ Skokové (přechodové) a impulzní charakteristiky}
\label{sec:ukol11}
Na obrázku \ref{fig:charakteristiky} jsou zobrazeny tři impulzní a tři přechodové charakteristiky pro tři různé systémy

\subsection{~}
Přiřaďte správně skokové charakteristiky impulzním charakteristikám tak, aby odpovídaly stejnému systému. \\
Řešení: S ohledem na \ref{sec:ukol11:2} lze k sobě přiřadit odezvy podle tabulky \ref{tab:odezvy}.
\begin{table}[htbp]
	\label{tab:odezvy}
	\centering
	\begin{tabular}{c|c|c}
		$h(t)$ & $w(t)$ & poznámka \\
		\hline
		c & e & integrál konstantní $h(t) \neq 0 $ roste lineárně a diverguje \\
		b & d & $h(t)$ i $w(t)$ exponeniciální průběh \\
		a & f & $h(t)$ kmitá, kmihy musíme najít i na $w(t)$
	\end{tabular}
	\caption{Přiřazení odezvy na Diracův impuls a na jednotkový skok}
\end{table} 

\subsection{~}
\label{sec:ukol11:2}
Jaký obecně mezi sebou mají impulzní charakteristika $h(t)$ a skoková charakteristika $w(t)$ vztah? \\
Řešení (viz \ref{sec:ukol1:3}): Protože spojitý lineární systém zachovává vztahy jako integrál, derivace a lineární kombinace,
je vztah $h(t)$ a $w(t)$ stejný jako vztah signálů $\delta(t)$ a $\mathbb{1}$(t). Platí
\begin{equation*}
	h(t) = \frac{\dif}{\dif t} w(t) ~~ a ~~~~~~~~~ w(t) = \int_0^{t} h(\tau) \dif \tau.
\end{equation*}


\begin{figure}[htbp]
    \centering % <-- added
\begin{subfigure}{0.25\textwidth}
  \includegraphics[width=\linewidth]{zadani11-a}
  \caption{}
  \label{fig:charakteristiky:a}
\end{subfigure}\hfil % <-- added
\begin{subfigure}{0.25\textwidth}
	\includegraphics[width=\linewidth]{zadani11-b}
	\caption{}
	\label{fig:charakteristiky:b}
\end{subfigure}\hfil % <-- added
\begin{subfigure}{0.25\textwidth}
	\includegraphics[width=\linewidth]{zadani11-c}
	\caption{}
  \label{fig:charakteristiky:c}
\end{subfigure}

\medskip
\begin{subfigure}{0.25\textwidth}
  \includegraphics[width=\linewidth]{zadani11-d}
  \caption{}
  \label{fig:charakteristiky:d}
\end{subfigure}\hfil % <-- added
\begin{subfigure}{0.25\textwidth}
	\includegraphics[width=\linewidth]{zadani11-e}
	\caption{}
	\label{fig:charakteristiky:e}
\end{subfigure}\hfil % <-- added
\begin{subfigure}{0.25\textwidth}
	\includegraphics[width=\linewidth]{zadani11-f}
  \label{fig:charakteristiky:f}
  \caption{}
\end{subfigure}
\caption{Charakteristiky k identifikaci}
\label{fig:charakteristiky}
\end{figure}


\section{Frekvenční charakteristiky}
\label{sec:ukol12}
Na obrázku \ref{fig:frekchar} jsou zobrazeny Bodeho frekvenční charakteristiky a Nyquistovy frekvenční charakteristiky tří systémů.
\subsection{~}
Přiřaďte k sobě frekvenční charakteristiky odpovidající stejnému systému. \\
Řešení: Při přiřazení charakteristik si musím vystačit s obecným trendem fáze a amplitudy, protože z Nyquistovy char. nelze odečíst přesná frekvence.
Amplitudy v této úloze nebyly příliš nápomocné, všechny charakteristiky lze určit pohledem na fáze. Jedna dvojice má fáze konstantní,
druhá dvojice mění fázi na celém spektru jen o 90° a poslední o 270°.
Charakteristiky lze přiřadit podle tabulky \ref{tab:charakteristiky}.
\begin{table}[htbp]
	\label{tab:charakteristiky}
	\centering
	\begin{tabular}{c|c|c}
		Bode & Nyquist & poznámka \\
		\hline
		c & e & jediná možnost s konstantní fází, amplitudy omezeny shora \\
		a & f & pro $\omega \rightarrow 0$ je fáze 0, celkem změna fáze o 90° \\
		b & d & celkem změna fáze o 270°
	\end{tabular}
	\caption{Přiřazení odpovídajícíh si frekvenčních charakteristik}
\end{table}
\subsection{~}
U frekvenční charakteristiky na obrázku \ref{fig:frekchar:f} najděte pro fázi -45° zesílení systému (v dB).


\begin{figure}[htbp]
    \centering % <-- added
\begin{subfigure}{0.25\textwidth}
  \includegraphics[width=\linewidth]{zadani12-a}
  \caption{}
  \label{fig:frekchar:a}
\end{subfigure}\hfil % <-- added
\begin{subfigure}{0.25\textwidth}
	\includegraphics[width=\linewidth]{zadani12-b}
	\caption{}
	\label{fig:frekchar:b}
\end{subfigure}\hfil % <-- added
\begin{subfigure}{0.25\textwidth}
	\includegraphics[width=\linewidth]{zadani12-c}
  \caption{}
  \label{fig:frekchar:c}
\end{subfigure}

\medskip
\begin{subfigure}{0.25\textwidth}
  \includegraphics[width=\linewidth]{zadani12-d}
  \caption{}
  \label{fig:frekchar:d}
\end{subfigure}\hfil % <-- added
\begin{subfigure}{0.25\textwidth}
	\includegraphics[width=\linewidth]{zadani12-e}
	\caption{}
	\label{fig:frekchar:e}
\end{subfigure}\hfil % <-- added
\begin{subfigure}{0.25\textwidth}
	\includegraphics[width=\linewidth]{zadani12-f}
  \caption{}
  \label{fig:frekchar:f}
\end{subfigure}
\caption{Frekvenční charakteristiky ke spojení}
\label{fig:frekchar}
\end{figure}




% ---------------------------------
% ---------------------------------
% Literatura
\begin{thebibliography}{9}

% vzorová citace
\bibitem{lamport94}
  Leslie Lamport,
  \emph{\LaTeX: A Document Preparation System}.
  Addison Wesley, Massachusetts,
  2nd Edition,
  1994.

\bibitem{Wiki}
	\LaTeX tutorials, \url{http://en.wikibooks.org/wiki/LaTeX/}

\bibitem{ARI11}
	Studenti předmětu ARI 2011, \emph{ARI song (videoklip)} \url{http://www.youtube.com/watch?v=5gDfQK7dD7c}
\end{thebibliography}












\end{document}

